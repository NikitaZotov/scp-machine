\begin{SCn}

\scnheader{Агент интерпретации оператора управления scp-процессами}
\scnidtf{sc-агент интерпретации оператора управления scp-процессами}
\scntext{примечание}{Первичным условием инициирования агента интерпретации оператора управления scp-процессами является появление выходящей константной постоянной sc-дуги принадлежности из класса active\_action.}
\begin{scnrelfromset}{реализованные операторы}
	\scnitem{scp-оператор завершения выполнения программы}
	\scnitem{scp-оператор ожидания завершения выполнения scp-программы}
	\scnitem{scp-оператор асинхронного вызова подпрограммы}
	\scnitem{scp-оператор ожидания события}
\end{scnrelfromset}
\begin{scnrelfromset}{нереализованные операторы}
	\scnitem{конъюнкция предшествующих scp-операторов}
	\scnitem{scp-оператор ожидания завершения выполнения множества scp-программ}
\end{scnrelfromset}
\begin{scnrelfromvector}{задачи}
	\scnfileitem{Интерпретация активного оператора управления scp-процессами.}
\end{scnrelfromvector}
\scnrelfrom{пример входной конструкции}{\scnfileimage[8em]{images/scp_process_control_operator_interpreter_input.png}}
\begin{scnrelfromset}{аргументы агента}
	\scnitem{пустое множество}
\end{scnrelfromset}
\scnrelfrom{ответ агента}{ответ агента интерпретации оператора управления scp-процессами}
\begin{scnindent}
	\scneq{пустое множество}
\end{scnindent}
\scnrelfrom{пример выходной конструкции}{\scnfileimage[30em]{images/scp_process_control_operator_interpreter_output.png}}

\end{SCn}

