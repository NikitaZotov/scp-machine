\begin{SCn}

\scnheader{Агент интерпретации оператора обработки содержимых файлов, связанной с математическими вычислениями}
\scnidtf{sc-агент интерпретации оператора обработки содержимых файлов, связанной с математическими вычислениями}
\scntext{примечание}{Первичным условием инициирования агента интерпретации оператора обработки содержимых файлов, связанной с математическими вычислениями является появление выходящей константной постоянной sc-дуги принадлежности из класса active\_action.}
\begin{scnrelfromset}{реализованные операторы}
	\scnitem{scp-оператор вычисления синуса числового содержимого файла}
	\scnitem{scp-оператор вычисления косинуса числового содержимого файла}
	\scnitem{scp-оператор вычисления тангенса числового содержимого файла}
	\scnitem{scp-оператор вычисления арксинуса числового содержимого файла}
	\scnitem{scp-оператор вычисления арккосинуса числового содержимого файла}
	\scnitem{scp-оператор вычисления арктангенса числового содержимого файла}
	\scnitem{scp-оператор нахождения остатка от деления числовых содержимых файлов}
	\scnitem{scp-оператор нахождения целой части от деления числовых содержимых файлов}
	\scnitem{scp-оператор деления числовых содержимых файлов}
	\scnitem{scp-оператор умножения числовых содержимых файлов}
	\scnitem{scp-оператор вычитания числовых содержимых файлов}
	\scnitem{scp-оператор сложения числовых содержимых файлов}
	\scnitem{scp-оператор вычисления логарифма числового содержимого файла}
	\scnitem{scp-оператор возведения числового содержимого файла в степень}
\end{scnrelfromset}
\begin{scnrelfromvector}{задачи}
	\scnfileitem{Интерпретация активного оператора обработки содержимых файлов, связанной с математическими вычислениями.}
\end{scnrelfromvector}
\scnrelfrom{пример входной конструкции}{\scnfileimage[8em]{images/scp_math_operator_interpreter_input.png}}
\begin{scnrelfromset}{аргументы агента}
	\scnitem{пустое множество}
\end{scnrelfromset}
\scnrelfrom{ответ агента}{ответ агента интерпретации оператора обработки содержимых файлов, связанной с математическими вычислениями}
\begin{scnindent}
	\scneq{пустое множество}
\end{scnindent}
\scnrelfrom{пример выходной конструкции}{\scnfileimage[30em]{images/scp_math_operator_interpreter_output.png}}

\end{SCn}

