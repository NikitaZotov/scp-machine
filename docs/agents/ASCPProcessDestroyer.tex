\begin{SCn}

\scnheader{Агент уничтожения scp-процессов}
\scnidtf{sc-агент уничтожения scp-процессов}
\scntext{примечание}{Первичным условием инициирования агента уничтожения scp-процессов является появление выходящей постоянной константной sc-дуги принадлежности из класса action\_finished.}
\begin{scnrelfromvector}{задача}
	\scnfileitem{Удалить все операнды и множества пар, являющиеся операндами с ролями rrel\_2 и rrel\_3 у операторов процесса, принадлежащих классам sys\_search и sys\_gen.}
	\scnfileitem{Удалить операнд с ролью rrel\_2 у операторов процесса, принадлежащих классу call.}
	\scnfileitem{Удалить все операторы процесса.}
	\scnfileitem{Удалить множество операторов процесса.}
	\scnfileitem{Удалить процесс.}
\end{scnrelfromvector}
\scnrelfrom{пример входной конструкции}{\scnfileimage[30em]{images/scp_process_destroyer_input.png}}
\begin{scnrelfromset}{аргументы агента}
	\scnitem{пустое множество}
\end{scnrelfromset}
\scnrelfrom{ответ агента}{ответ агента уничтожения scp-процессов}
\begin{scnindent}
	\scneq{пустое множество}
\end{scnindent}
\scnrelfrom{пример выходной конструкции}{\scnfileimage[30em]{images/scp_process_destroyer_output.png}}

\end{SCn}
