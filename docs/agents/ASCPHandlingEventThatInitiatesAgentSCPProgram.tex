\begin{SCn}

\scnheader{Агент обработки события в sc-памяти, инициирующего агентную scp-программу}
\scnidtf{sc-агент обработки события в sc-памяти, инициирующего агентную scp-программу}
\scntext{примечание}{Первичным условием инициирования агента обработки события в sc-памяти, инициирующего агентную scp-программу, является элементарное sc-событие, указанное в первичном условии инициировния абстрактного sc-агента, агентная scp-программа которого инициируется.}
\scntext{примечание}{Агент обработки события в sc-памяти, инициирующего агентную scp-программу, используется для осуществления логики scp-агентов: реагирование на первичное условие инициирования и запуск интерпретации scp-программы.}
\begin{scnrelfromvector}{задачи}
	\scnfileitem{Среагировать на первичное условие инициирования абстрактного sc-агента.}
	\scnfileitem{Найти в базе знаний scp-программу, указанную во множестве программ соответствующего абстрактного sc-агента.}
	\scnfileitem{Начать выполнение действия класса action\_scp\_interpretation\_request, указав в качестве аргументов действия найденную scp-программу и ориентированное множество аргументов этой scp-программы, состоящее из двух элементов: scp-программы и sc-дуги, с которой связано первичное условие инициирования.}
\end{scnrelfromvector}
\scnrelfrom{пример входной конструкции}{\scnfileimage[8em]{images/scp_interpretation_request_initiation_input.png}}
\begin{scnindent}
	\scntext{пояснение}{Первичным условием инициирования агента обработки события в sc-памяти, инициирующего агентную scp-программу, может быть любое элементарное sc-событие.}
\end{scnindent}
\begin{scnrelfromset}{аргументы агента}
	\scnitem{пустое множество}
\end{scnrelfromset}
\scnrelfrom{ответ агента}{ответ агента обработки события в sc-памяти, инициирующего агентную scp-программу}
\begin{scnindent}
	\scntext{примечание}{В результате выполнения агентом результат действия никак не меняется, в результате находятся только те элементы, которые были в него добавлены в процессе интерпретации scp-программы.}
\end{scnindent}
\scnrelfrom{пример выходной конструкции}{\scnfileimage[30em]{images/scp_interpretation_request_initiation_output.png}}

\end{SCn}
